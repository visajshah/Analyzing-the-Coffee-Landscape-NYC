\documentclass{article}
\usepackage[utf8]{inputenc}
\usepackage{hyperref}

\title{\textbf{Analyzing the Coffee Landscape: \\\textit{NYC} and \textit{Toronto}}}
\author{Author: Visaj Nirav Shah}
\date{\emph{September 2020}}

\begin{document}

\maketitle

\section{Problem Statement}
{Analyze neighborhood clusters of \textit{New York City (NYC), USA} and \textit{Toronto, Canada} based on the number of coffeehouses and cafes present and the ratings of the stores.}
\subsection{Purpose}
{\textit{NYC} and \textit{Toronto} are financial capitals of their respective countries, popular tourist destinations, have a huge population, and house offices of important international institutions and major corporate headquarters. In such a landscape, coffeehouses are a crucial part of everyday life. People socialize in coffee shops, hold informal group meetings, relax after a long day, and so on. Hence, it is imperative to analyze the coffee scene in such metros where millions depend on a cup of fresh brew.}
\subsection{Target Audience}
{The results of this study will be of particular interest to the major stakeholders mentioned below:-}
\begin{itemize}
    \item \textbf{Major Coffeehouse and Cafe Chains}\\{Corporate chains working in this field can use the results to predict the market opportunities and future growth for their firm. This study will help them select the most profitable next new location.}
    \item \textbf{Prospective Franchisees and Coffeehouse Owners}\\{People wanting to buy franchise stores of chains, like \textit{Starbucks}, \textit{Tim Hortons}, and so on, can understand the landscape and spread of quality stores. This will help them enter the market in a profitable manner with an idea of the competition and target customers.}
    \pagebreak
    \item \textbf{Offices and Public Places}\\{Corporate offices, especially small and medium businesses, need to ensure that their business location is attractive enough to be inviting to customers. Employees and visitors should know the options of going outside for informal meetings, relaxing, and so on.}
    \item \textbf{Tourists and Residents}\\{Tourism is a substantial part of the economy of both the cities. There is a constant and large influx of tourists who visit this city. This project can help the tourists make an informed choice about deciding where to enjoy a cup of refreshing coffee. Similarly, people new to the city or neighborhood can understand the surrounding coffee scene better.}
\end{itemize}
\section{Data}
\textbf{\textit{Places API} - \textit{Foursquare Developers}}\\{\textit{Foursquare Labs Inc.} is a technology product company based in \textit{New York City, USA}. Their primary product is a local search-and-discovery platform. Users can record their check-ins, review everyday places, and add other useful details, experiences, and so on about places they visit.}\\\\{Foursquare collects the information provided by users, and enables developers access the same for development purposes. The information can be retrieved using \textit{Foursquare Developers' Places API}\footnote{Foursquare Developers' Places API:- \url{https://developer.foursquare.com/docs/places-api/}}.}\\\\{The available data is reach with many features. Primarily, for our problem statement, we will use the count of coffeehouses and cafes in a neighborhood and the ratings of locations. For example, number of coffee shops in \textit{Upper East Side, NYC} and the ratings of these locations.}
\subsection{\textit{New York City, New York, USA}}
\begin{itemize}
    \item \textbf{\textit{2014 New York City Neighborhood Names}}\\{This \textit{New York City Neighborhood Names} point file was created as a guide to \textit{New York City}’s neighborhoods that appear on the web resource, \textit{“New York: A City of Neighborhoods.”} Best estimates of label centroids were established at a 1:1,000 scale, but are ideally viewed at a 1:50,000 scale.\footnote{2014 New York City Neighborhood Names:- \url{https://geo.nyu.edu/catalog/nyu\_2451\_34572}} From this dataset, we are going to get the list of each neighborhood in \textit{NYC}.}\\\\\\\\{This dataset, published by \textit{Department of City Planning, New York (City)} in 2014, is held by \textit{New York University (NYU)} and is available on \textit{NYU Spatial Data Repository}.}
\end{itemize}
\subsection{\textit{Toronto, Ontario, Canada}}
\begin{itemize}
    \item \textbf{\textit{List of postal codes of Canada: M - Wikipedia}}\\{Data of \textit{Toronto}'s neighborhoods is collected by web-scraping this \textit{Wikipedia} webpage\footnote{List of postal codes of Canada: M:- \url{https://en.wikipedia.org/wiki/List_of_postal_codes_of_Canada:_M}}. Web-scraping was done using the \textit{Python} library \textit{beautifulsoup4}. From this dataset, we are going to get the list of each neighborhood in \textit{Toronto}. }\\\\{References for this page include the official \textit{Canada Post} data and \textit{Statistics Canada} data. Both of these organizations function under the \textit{Govt. of Canada}.}
\end{itemize}
%\section{Methodology}
%\section{Results}
%\section{Discussion}
%\section{Conclusion}
\end{document}
